\usepackage{lipsum}
\usepackage{listings}
\usepackage{hyperref}% !TeX root = ..\main.tex

\section{\tbd{What are Graph based Databases?}}
Briefly, a graph based database is a combination of edges and vertices which  can be used to represent real world
objects and their relationships.
It stores data in form of vertices wich are called nodes and edges wich are called relationships.
Nodes and relationships can be connected to each other in a graph.
The nodes are the objects and the relationships are the connections between the objects.
Nodes and relationships can habe properties.
The properties can be used to store metadata about the node or relationship.\cite[P. 6f. ]{PractivalNeo4j}
You can think of a graph based database as a network.

A good example of a real life graph based database is a crime diagram.
In a crime diagram you can see the different people and their relationships to each other.
The people are the nodes and the relationships are the connections between the people. \cite[P. 6f. ]{BeginningNeo4j}
%toDO: add picture

\subsection{\tbd{Nodes}}
Nodes can be seen as the objects in a graph based database.
They can be tagged with labels\ref{subsec:labels} to describe their different roles and tasks in the graph model.
Nodes can also hold an arbitrary number of properties\ref{subsec:tbd{properties}}.
The following example shows two nodes with the labe \("\)Person\("\), the properties \("\)name\("\), \("\)lastname\("\)
and \("\)email\("\) and their relationship to each other.
%todo: add picture and make the graphmodel.
\subsection{\tbd{Relationships}}\label{subsec:tbd{relationships}}
\subsection{\tbd{Properties}} \label{subsec:tbd{properties}}
\subsection{\tbd{Labels}} \label{subsec:labels}
\subsection{\tbd{Traversal and indexing}}\label{subsec:tbd{traversal-and-indexing}}
