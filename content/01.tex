% !TeX root = ..\main.tex

\todo{Change}
\enquote{This is a test.} \parencite{test}

\textcite{test} writes tests.

\lipsum[1-2]

This is a example Image:

\begin{figure}
    \centering
    \includegraphics[width=0.7 \linewidth]{images/Example.png}
    \caption{This is a test figure caption}
\end{figure}