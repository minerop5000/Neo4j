\section{Conclusion and recommendations}
\subsection{Conclusion}
\subsubsection{Graph databases in general}
Graph databases are a good have a very different data model than relational
databases. Because of this a lot of common rules and best practices do not apply
to graph databases. This leads to some advantages and disadvantages of graph
databases. Probably the biggest advantage of graph databases is their
performance with deeply nested data. On the other hand their performance lacks
in traditional relational data.

\subsubsection{Neo4j}
Neo4j is not without reasons the most popular graph database
\parencite{db-engines}. As our brief history showed
Neo4j is a very fast developing database. While there are some basic
similarities between Cypher and SQL, like the use of \texttt{WHERE} clauses,
both query languages are mostly very different. The main difference is the graph
like structure of the data. This leads to a very different way of thinking about
the queries. While Cypher may be different to the more common SQL, after a short
time it is very easy to get used to it. Another advantage of Neo4j is the great
visualization of the data.

\subsection{Recommendations}
As mentioned above Cypher is very different to SQL, because of that we would
recommend you to get familiar with the query language before you start using it
in production. The official documentation is a good place to start learning the
language \parencite{neo4j:cypher-manual}.

We wouldn't recommend replacing every relational database with a Neo4j.
Instead there are three main use cases for graph databases:
\begin{itemize}
      \item \textbf{Easy setup and debugging:} Especially with the official docker
            image Neo4j is easy to set up. The integrated Neo4j web interface is a good tool for
            debugging the database.
      \item \textbf{Visualization:} Neo4j is a good choice for visualization of data
            because of the graph structure. In comparison to relational databases the
            visualization of data is much easier. The graph like structure of the data is
            far more intuitive than the table structure of relational databases.
      \item \textbf{Deeply nested data:} Graph databases in general are a good choice
            for deeply nested data. In these cases they outperform relational databases in
            terms of performance and simplicity.
\end{itemize}
