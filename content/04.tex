% !TeX root = ..\main.tex

\section{Brief History of Neo4j} \label{sec:history}
Neo4j was founded in 2007 by Emil Eifrem and Johan Oskarsson and is written in Java.
However, the fist usable version of Neo4j was released in 2010.
It was the birth of the graph based Databases with nodes and relationships.
The first version of Neo4j was called Neo4j Community Edition.
Over the years, Neo4j has been developed and improved.
In the following list you can see the different versions of Neo4j, their pros and improvements and their release dates .
\textbf{Neo4j 1.0} Was released 2010
\begin{itemize}
    \item It was the birth of graph based databases with nodes and relationships \parencite{Neo4j:V1}.
\end{itemize}
\textbf{Neo4j 2.0} Was released 2013.
\begin{itemize}
    \item First version of Neo4j with a Cypher Query Language.
    \item It gave Neo4j a better interface.
    \item Addition of a REST API \parencite{Neo4j:V2}.
\end{itemize}
\textbf{Neo4j 3.0} Was released 2016.
\begin{itemize}
    \item Higher upper scaling limits.
    \item Faster access to data-graphs.
    \item Access to frameworks and Web interfaces \parencite{Neo4j:V3}.
\end{itemize}
\textbf{Neo4j 4.0} Was released 2019.
\begin{itemize}
    \item Availability of multiple database at runtime in standalone.
    \item Schema-based security and Role Based Access Control.
    \item Memory constrain control for transactions \parencite{Neo4j:V4}.
\end{itemize}
\textbf{Neo4j 4.4} Was released 2021.
\begin{itemize}
    \item Long term support for enterprise.
    \item Cypher Shell Enhancements.
    \item HTTP API with cloud-native interface \parencite{Neo4j:V4.4}.
\end{itemize}
\textbf{Neo4j 5.5} Was released 2023.
\begin{itemize}
    \item Current version at the time of this paper.
    \item New community Libraries.
    \item More visualization tools
    \item And much more \parencite{Neo4j:V5}.
\end{itemize}
