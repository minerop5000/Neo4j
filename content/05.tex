% !TeX root = ..\main.tex

\section{Pros and Cons of graph based databases}\label{sec:pros-and-cons-of-graph-based-databases}
In this section we will discuss the pros and cons of graph based databases.
\subsection{Pros of a graph based database}\label{subsec:pros-of-a-graph-based-database}\label{subsec:pros}
\begin{itemize}
    \item \textbf{Good for relationships} Modeling of relations is easy in a graph based database because you can implement them as you imaging them
    \item \textbf{Easy to scale} Another advantage of a graph based database is that the performance of a graph based database is not affected by the number of nodes and relationships so it can grow year over year without any performance issues.
    \item \textbf{Flexibility} The structure of graph based databases are easy to edit and easy to extend without endangering the current functionality.
    \item \textbf{Agile} The last named advantage provides another big advantage of graph based databases, their flexibility leads to an advantage in Agile-teams, because the graph based database can grow with the tasks and goals of them.
    \item \textbf{You get what you see} Last but not least, in a graph based database you can see the functionality and relations between nodes by an instance without even thinking about it.
\end{itemize}
\subsection{Cons of a graph based database}\label{subsec:cons-of-a-graph-based-database}\label{subsec:cons}
\begin{itemize}
    \item \textbf{Not often used} Graph based databases are used much less then traditional databases. Therefore it is somtimes difficult to get support, help or anaswers to certain questions.
    \item \textbf{Slower} Graph based databases are significantly slower in processing transactional.
    \item  \textbf{Security} They also have no security on the data level.
\end{itemize}

%todo: add cites from powerpoint